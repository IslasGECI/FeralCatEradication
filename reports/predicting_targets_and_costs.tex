\documentclass{article} % El documento es de tipo artículo
\usepackage[utf8]{inputenc} % Paquete que permite escribir caracteres especiales
\usepackage[spanish, es-tabla]{babel} % Paquete para cambiar "Cuadro" a "Tabla" en encabezados de tablas
\usepackage{graphicx} % Paquete para importar figuras
\usepackage{hyperref} % Paquete para agregar vínculos como enlaces
\usepackage{booktabs}
\usepackage{authblk}
\usepackage{amsmath}
\usepackage{float}
\usepackage{xcolor}
\usepackage{caption}
\usepackage{csvsimple}
\usepackage{pythontex}
\usepackage{pgfplotstable}
\usepackage{adjustbox}
\usepackage{pdflscape} 
\usepackage[a4paper,top=2cm,bottom=2cm,left=3cm,right=3cm,marginparwidth=1.75cm]{geometry}

\author{Braulio Rojas, Maritza Bello, Xavier Paredes y Fernando Alvarez}


\title{Predicción de costos y objetivos para la reducción de gato feral en grandes islas utilizando modelos estocásticos de población \\ \begin{large} Grupo de Ecología y Conservación de Islas \end{large}}

\begin{document}

\maketitle

\begin{abstract}
\end{abstract}


\begin{figure}[H]
    \centering
\includegraphics[scale=0.6]{figures/reduction_factor.jpg}
\caption{Factor de reducción}
\label{fig:reductionFactor}
\end{figure}

\begin{figure}[H]
    \centering
\includegraphics[scale=0.6]{figures/simulation.jpg}
\caption{Proporción promedio de la población inicial de gatos (N1) en Isla Canguro proyectada para un marco de tiempo de 10 años para el escenario sin control. La línea negra indica el valor de la mediana con 10,000 iteraciones, junto con intervalos de confianza del 95% (área sombreada en gris).}
\label{fig:simulation}
\end{figure}

\begin{figure}[H]
    \centering
\includegraphics[scale=0.6]{figures/constant_proportional_annual_cull.jpg}
\caption{Proporción mínima de la población de gatos salvajes de la Isla Canguro que queda después de un sacrificio anual proporcional constante que oscila entre 0,2 y 0,9. La línea negra continua representa la proporción mínima mediana de la población inicial (N1) después de 10,000 iteraciones con intervalos de confianza del 95% indicados como área sombreada en gris}
\label{fig:constantProportionalAnnualCull}
\end{figure}


\end{document}