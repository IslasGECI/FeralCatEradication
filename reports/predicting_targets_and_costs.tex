\documentclass{article} % El documento es de tipo artículo
\usepackage[utf8]{inputenc} % Paquete que permite escribir caracteres especiales
\usepackage[spanish, es-tabla]{babel} % Paquete para cambiar "Cuadro" a "Tabla" en encabezados de tablas
\usepackage{graphicx} % Paquete para importar figuras
\usepackage{hyperref} % Paquete para agregar vínculos como enlaces
\usepackage{booktabs}
\usepackage{authblk}
\usepackage{amsmath}
\usepackage{float}
\usepackage{xcolor}
\usepackage{caption}
\usepackage{csvsimple}
\usepackage{pythontex}
\usepackage{pgfplotstable}
\usepackage{adjustbox}
\usepackage{pdflscape} 
\usepackage[a4paper,top=2cm,bottom=2cm,left=3cm,right=3cm,marginparwidth=1.75cm]{geometry}

\author{Braulio Rojas, Maritza Bello, Xavier Paredes y Fernando Alvarez}


\title{Predicción de costos y objetivos para la reducción de gato feral en grandes islas utilizando
modelos estocásticos de población \\ \begin{large} Grupo de Ecología y Conservación de Islas
\end{large}}

\begin{document}

\maketitle

\begin{abstract}

Reprodujimos las gráficas del artículo "Predicting targets and costs for feral-cat reduction on
large islands using stochastic population models" \cite{venning2021predicting}.

\end{abstract}


\begin{figure}[H]
    \centering
\includegraphics[scale=0.6]{figures/reduction_factor.jpg}
\caption{Factor de reducción de gato feral en grandes islas.}
\label{fig:reductionFactor}
\end{figure}

\begin{figure}[H]
    \centering
\includegraphics[scale=0.6]{figures/simulation.jpg}
\caption{Proporción promedio de la población inicial ($N_1$) de gato feral (N1) en Isla Canguro proyectada para
un periodo de tiempo de 10 años para el escenario sin control.}
\label{fig:simulation}
\end{figure}

\begin{figure}[H]
    \centering
\includegraphics[scale=0.6]{figures/constant_proportional_annual_cull.jpg}
\caption{Proporción ($N_1$) mínima de la población remanente de gato feral de Isla Canguro después de
	un sacrificio anual constante. Los valores de la proporción del sacrificio anual cambian en cada
    iteración de la simulación. Los valores son proporcional al tamaño inicial de la población, y 
    barren los valoras entre 0.2 y 0.9.}
\label{fig:constantProportionalAnnualCull}
\end{figure}

\begin{figure}[H]
    \centering
\includegraphics[scale=0.6]{figures/monthly_time_serie_individuals.jpg}
\caption{Proporción del tamaño poblacional ($N_1$) promedio de gato feral en Isla Canguro. El tamaño
es proyectado para un periodo de tiempo de 120 meses.}
\label{fig:monthlyTimeSerieIndividuals}
\end{figure}

\begin{figure}[H]
\centering
\includegraphics[scale=0.6]{figures/culling_contour_plot.png}
\caption{Proporción del tamaño de la población ($N_1$) al final de un proyecto de erradicación. El
eje x contiene distintos valores de la proporción de gato feral sacrificado para lo dos primeros
años del proyecto. El eje y representa la proporción de gato feral sacrificado para el resto del
proyecto.}
\label{fig:culling_contour_plot}
\end{figure}

\begin{figure}[H]
\centering
\includegraphics[scale=0.6]{figures/cost_hunt_contour_plot.png}
\caption{Costo (AU\$) del proyecto de erradicación de gato feral en Isla Canguro considerando la
cacería como método principal.}
\label{fig:cost_hunting_contour_plot}
\end{figure}

\begin{figure}[H]
\centering
\includegraphics[scale=0.6]{figures/cost_felixer_contour_plot.png}
\caption{Costo (AU\$) del proyecto de erradicación de gato feral en Isla Canguro considerando
trampas Felixer como método principal.}
\label{fig:cost_felixer_plot}
\end{figure}

\begin{figure}[H]
\centering
\includegraphics[scale=0.6]{figures/cost_traps_contour_plot.png}
\caption{Costo (AU\$) del proyecto de erradicación de gato feral en Isla Canguro considerando
trampas cepo como método principal.}
\label{fig:cost_tramps_contour_plot}
\end{figure}

\bibliography{../references/references}
\bibliographystyle{apalike}

\end{document}